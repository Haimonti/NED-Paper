\section{Problem definition:}
The task of named entity disambiguation can be formalised as follows:
An entity name mentioned in the knowledge base may refer to multiple entities. Named entity disambiguation is the task of resolving the potential ambiguity among these entity mentions by mapping an entity name to a representation of the entity in a knowledge base like historical newspapers. Given an entity-name mention n in an unstructured text corpus C, the goal is to disambiguate the occurrences of n in C using some word representation W such that each n can be associated with an entity E in C. 

\section{Methodology - Named Entity Disambiguation using LDA}
The goal of named entity disambiguation is to find and map each mention of an entity name to an entity in the knowledge base by exploiting the context surrounding the mention. The reasoning behind using the context is that it provides a set of representative words which can help distinguish between entities in the knowledge base. Therefore, our model uses the context - specifically the profession of an entity mention for disambiguation. Each profession is modelled as a topic/label on the context of an entity mention given by x words before and x words after an entity name mention.  In a large text corpus, it is reasonable to make an assumption that multiple occurrences of an entity mention in any given document refers to the same entity. Thus, we only consider multiple occurrences of entity name mentions across documents. Each document can be associated with only one topic. 

Although the context size is limited to ensure topic centrality, some words may not pertain to any one topic or be generic to some or all the topics. In this case, the topic cannot be clearly ascertained and therefore a special topic called "unclear" is assigned. The document - label mapping and the label-word mapping can be modelled based on the above intuitions. Figure a. provides an overview of our proposed model. A custom profession dictionary that consists of a job-title and the words related to the job-title is used for topic association.
