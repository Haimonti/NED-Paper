\def\year{2018}\relax
%File: formatting-instruction.tex
\documentclass[letterpaper]{article} %DO NOT CHANGE THIS
\usepackage{aaai18}  %Required
\usepackage{times}  %Required
\usepackage{helvet}  %Required
\usepackage{courier}  %Required
\usepackage{url}  %Required
\usepackage{cite}
\usepackage{graphicx}  %Required
\frenchspacing  %Required
\setlength{\pdfpagewidth}{8.5in}  %Required
\setlength{\pdfpageheight}{11in}  %Required
%PDF Info Is Required:
  \pdfinfo{
/Title (2018 Formatting Instructions for Authors Using LaTeX)
/Author (AAAI Press Staff)}
\setcounter{secnumdepth}{0}  
 \begin{document}
% The file aaai.sty is the style file for AAAI Press 
% proceedings, working notes, and technical reports.
%
\title{Named Entity Disambiguation on a Historical Newspaper Archive}
\author{  Haimonti Dutta, Jayashree Chandrasekaran and Neha Lalaso Jagtap\\
Department of Management Science and Systems\\
Department of Computer Science \\
University at Buffalo, NY 14221\\
\{haimonti, jchandra, njagtap2\}@buffalo.edu\\
}
\maketitle
\begin{abstract}
Named Entity Disambiguation (NED) is the task of disambiguating entity mentions in natural language text. Often, the task of disambiguation involves generating evidences (such as textual and structural features) about the named entity from the text and linking the information with entries in a knowledge base. In this paper, we present a generative model based on Latent Dirichlet Allocation (LDA) to disambiguate person names occurring in noisy text. Each entity appears in the context of representative words which can be used for disambiguation -  specifically, each entity is assumed to be a topic and the representative words are generated from the topic; therefore, each entity's context can be modeled as a document which is associated with a unique label. The mixture of topics can then be interpreted for the task of disambiguation. Empirical results presented on a corpus of 14020 historical newspaper articles show that this generative model can obtain reasonable accuracy in the task of Named Entity Disambiguation. 
\end{abstract}


\section{Introduction}

\section{Related Work}
Name entity disambiguation is popular area of research. But when it comes to the NED in noisy text, definition of noisy data is limited to the free text unstructured data. Most of the studies use Wikipedia as their knowledge base. ~\cite{cucerzan2007large} realized the effectiveness of using topical coherence to help named entity disambiguation. They used overlaps in categories and incoming links in Wikipedia to calculate the topical coherence between the referent entity candidate and other entities within the same context. Furthermore \cite{kataria2011entit} proposed the weakly semi-supervised hierarchical topic model called Wikipedia-based Pachinko Allocation Model (WPAM ) for disambiguating entities. WPAM uses all the words in a document, including those in the vicinity of un-annotated references, to learn high-quality word-entity associations They used Wikipedia annotations to appropriately bias the assignment of entity labels to annotated words (and un-annotated words co-occurring with them), and the Wikipedia category hierarchy to capture entity context and co-occurrence patterns in a single unified disambiguation framework. \cite{sen2012collective} adopted a latent topic model to learn the context-entity association to help disambiguation.
\cite{bhattacharya2006latent} developed a probabilistic generative model for collectively resolving entities in relational data. Their model may be viewed as extending the Dirichlet process mixture model to capture relations between entities or components. They modeled the topical coherence as the association of an entity and the latent topics of a document \cite{pilz2011names} approached the problem of name ambiguity using thematic distances over describing documents. Their approach relies on semantic topics provided by LDA and is thus able to exploit more information than other, rather restrictive, word-matching. Methods. \cite{glaser2016named} proposed NED to disambiguate unknown persons without any specific textual training data. Instead, it uses some aggregate of the textual material about different people that share the same properties (e.g., nationalities and professions). Then they applied topic modeling on the data to obtain topic information. To disambiguate a new unknown person in a text document, they obtained topic information from the context of the person, then compared this topic information with the information from the extracted material to find out which properties are closest to the person. \cite{Li:2013} proposed a generative model and an incremental algorithm to automatically mine useful evidences across documents. They used topic modeling for getting background topic and unknown entities.


\section{Problem Definition}

Named Entity Disambiguation (NED) is the task of disambiguating entity mentions in natural language text. Often, the task of disambiguation involves generating evidences (such as textual and structural features) about the named entity from the text and matching or linking the information with corresponding entries in knowledge bases such as Wikipedia. The task is challenging because named entity mentions can be ambiguous - for example, the name ``Eugene Kelly" could refer to a banker in the city, a pastor in a local church or some other person.  
\emph{Named Entity Disambiguation: } Named Entity Disambiguation (NED) 


\section{Methodology - Named Entity Disambiguation using LDA}


\input{results.tex}

\section{The Data}
%\section{Data}


The Named Entity Disambiguation is performed on articles of historical newspapers obtained from Chronicling America\footnote{\texttt{http://chroniclingamerica.loc.gov/}}. The newspapers are scanned on a page-by-page basis and article level
segmentation is performed by human annotators. Articles of ``The Sun" newspaper from November-December 1894 consisting of 14020 news articles are used in our study. 
The Optical Character Recognition (OCR) scanning process used to convert historical newspaper pages into machine recognizable text is far
from perfect and the documents generated from it contain a large
amount of garbled text.
An individual OCR text article has at least one or more of the following types of spelling errors:
 \textbf{1) Real word errors} include words that are spelled correctly in the OCR text but still incorrect when compared to the original newspaper article image.  \textbf{2) Non-real word errors} include words that have been misspelled due to some insertion, deletion, substitution or transposition of characters from a word.  \textbf{3) Non-word errors} include words that have been spelled incorrectly and are a combination of alphabets and numerical characters. \textbf{4) New Line errors} include words that are separated by hyphens where part of a word is written on one text line and remaining part in the next line. 
\textbf{5) Word Split and Join errors} include words that either get split into one or more parts or some words in a sentence get joined to a make a single word. 
For example, in Figure~\ref{figure:1} the word ``coil"  has been correctly spelled in the OCR text but should have been ``and" according to the original newspaper article (real word error); the word ``tnenty" in the OCR text has a substitution error (`n' should have been `w') which is actually ``twenty" according to the original newspaper article (Non-real word error); the word ``4anrliteii" is a combination of alphabets and number and should have been ``confident" as per the original newspaper article(Non-word error); the word ``ex-ceptionally" where ``ex" occurs on one line while ``ceptionally" in the next and due to no punctuation in the text, they are treated as separate words in OCR text(New Line error); the word ``Thernndldntesnra" in the OCR text is actually a combination of three words ``The candidates are" while the words ``v Icrory" are actually equivalent to a single word ``victory" when compared with the original news article(Word Split and Join error).

%This is an
%initiative of the National Endowment for Humanities (NEH) and the
%Library of Congress (LC) whose goal is to develop an Internet-based, searchable database of U.S. newspapers(between
%1836 and 1922) with descriptive information and select digitization of historic pages. 
%Under this program, institutions such as libraries receive an award to select and digitize approximately 100,000 newspaper pages representing that state's regional history, geographic coverage, and events of the particular time period being covered. The scanned newspaper holdings of the New York Public Library are a source of prosopographical studies. 

\begin{figure*}
\includegraphics[scale=0.75]{originalimage}
\includegraphics[scale=0.80]{ocr}
\caption{Scanned Image of a Newspaper article (left) and its OCR raw text (right)}
\vspace{-10pt}
\label{figure:1}

\end{figure*}

%\subsection{Characteristics}


%\begin{itemize}
 %\item \textbf{Real word errors}	 include words that are spelled correctly in the OCR text but still incorrect when compared to the original newspaper article image. For example: In Figure~\ref{figure:1}, the word ``coil"  has been correctly spelled in the OCR text  but should have been ``and" according to the original newspaper article. 
 %\item \textbf{Non-real word errors} include words that have been misspelled due to some insertion, deletion, substitution or transposition of characters from a word. For eg. In Figure~\ref{figure:1}, the word ``tnenty" in the OCR text has a substitution error (`n' should have been `w') which is actually ``twenty" according to the original newspaper article.
 %\item \textbf{Non-word errors} include words that have been spelled incorrectly and are a combination of alphabets and numerical characters. For example: In Figure~\ref{figure:1}, the word ``4anrliteii" is a combination of alphabets and number and should have been ``confident" as per the original newspaper article.
%\item \textbf{New Line errors} include words that are separated by hyphens where part of a word is written on one text line and remaining part in the next line. For example: In Figure~\ref{figure:1}, the word ``ex-ceptionally" where ``ex" occurs on one line while ``ceptionally" in the next and due to no punctuation in the text, they are treated as separate words in OCR text.
%\item \textbf{Word Split and Join errors} include words that either get split into one or more parts or some words in a sentence get joined to a make a single word. For example: In Figure~\ref{figure:1}, the word ``Thernndldntesnra" in the OCR text is actually a combination of three words ``The candidates are" while the words ``v Icrory" are actually equivalent to a single word ``victory" when compared with the original news article.
%\end{itemize} 

%\subsection{Statistics}
%The OCR text available from Chronicling America website is on a page by page level and no article level segmentation is provided. OCR text dataset is therefore, taken from a PostgreSQL database where 
%A total of 8,403,844 tokens are generated from a bag-of-words extraction. 
%The newspaper database ER diagram \footnote{https://power.ldeo.columbia.edu/twiki/pub/Incubator/BodhiDBDesign/Final ERD.pdf }
%is used to extract the required articles text from the database by dumping complete dataset and extracting individual articles linetext based on their unique ID. 
%The text from the articles do not have any punctuation and contain a large amount of garbled text containing above mentioned OCR errors.




\section{Conclusions and Future Work}

\section{Acknowledgements}


\bibliography{aaai}
\bibliographystyle{aaai18}
\end{document}
